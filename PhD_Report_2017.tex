\documentclass[a4paper]{article}

\usepackage{authblk}

\usepackage{hyperref}

\hypersetup{
  pdfauthor = {Nikolaos Pothitos},
  pdftitle = {Annual Report for the PhD Thesis "Constraint
              Programming: Algorithms and Systems"}
}

\begin{document}

\title{Annual Report for the PhD Thesis ``Constraint
       Programming: Algorithms and Systems''}

\author{Nikolaos Pothitos ($\Delta$541)}

\affil{\normalsize
       Department of Informatics and Telecommunications \\
       National and Kapodistrian University of Athens \\
       Panepistimiopolis, 157\,84 Athens, Greece \\
       \texttt{pothitos@di.uoa.gr}}

\date{Academic Year 2016--2017}

\maketitle

\begin{itemize}
  \item \textsc{Naxos Solver} is a C++ Constraint
        Programming library.
  \item It is freely available in the open source community
        together with its full documentation~\cite{Naxos}.
  \item It enforces a bounds-consistency variant in order to
        propagate changes that happen in Constraint
        Satisfaction Problem variables~\cite{Bessiere2006}.
        This happens in every single node of the search
        tree.
  \item It supports the definition of custom search methods
        via a declarative API~\cite{Pothitos2017}.
  \item Declarativity is a \textsc{Naxos} key feature and it
        has do with describing problems and search methods
        independently. \textsc{Naxos} serves the Constraint
        Programming principle: \emph{The user states the
        problem, the computer solves it}~\cite{Freuder2014}.
  \item \textsc{Naxos} is being used in various
        applications, such as the construction of the
        timetable for the Department of Informatics and
        Telecommunications of the University of Athens every
        semester~\cite{Pothitos2012-Scheduling}.
\end{itemize}

\section*{Acknowledgments}

I would like to thank Prof.~Panagiotis Stamatopoulos for his
inspiration to build the solver and his continuous
supervision and scientific support. I also thank Foivos
Theocharis for the search methods modules implementation.

\bibliographystyle{abbrv}
\bibliography{bibliography}

\end{document}
